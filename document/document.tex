\documentclass[12pt,a4paper,onecolumn]{article}

%%%%%%%%%%%%%%%%%%%%%%%%%%%%%%%%%%%
%                 PACKAGES                  %
%%%%%%%%%%%%%%%%%%%%%%%%%%%%%%%%%%%

\usepackage[margin=1in]{geometry}
\usepackage{authblk}

% CODIFICACIÓN Y SOPORTE PARA ESPAÑOL
\usepackage[utf8]{inputenc}  % UTF-8 evita problemas de caracteres
\usepackage[T1]{fontenc}     % Mejor soporte de fuentes en LaTeX
\usepackage[spanish]{babel}  % Manejo correcto de idioma español

\usepackage{amsfonts}
\usepackage{graphicx}
\usepackage{xcolor}  % Ya incluye "color", por lo que eliminamos "color.sty"
\usepackage{amsmath}
\usepackage{amssymb}
\usepackage[table]{xcolor}
\usepackage{setspace}
\usepackage{booktabs}
\usepackage{dcolumn}
\usepackage{rotating}
\usepackage{threeparttable}
\usepackage[capposition=top]{floatrow}
\usepackage[labelsep=period]{caption}
\usepackage{subcaption}
\usepackage{lscape}
\usepackage{pdflscape}
\usepackage{multicol}
\usepackage[bottom]{footmisc}
\usepackage{longtable}
\usepackage{enumerate}
\usepackage{units}
\usepackage{placeins}
\usepackage{booktabs,multirow}

% Bibliografía
\usepackage{natbib}
\bibliographystyle{apalike}
\bibpunct{(}{)}{,}{a}{,}{,}

% Formato de párrafos
\renewcommand{\baselinestretch}{1}

% Definir columnas para tablas
\usepackage{array}
\newcolumntype{L}[1]{>{\raggedright\let\newline\\\arraybackslash\hspace{0pt}}m{#1}}
\newcolumntype{C}[1]{>{\centering\let\newline\\\arraybackslash\hspace{0pt}}m{#1}}
\newcolumntype{R}[1]{>{\raggedleft\let\newline\\\arraybackslash\hspace{0pt}}m{#1}}

% \usepackage{comment}  % Por ahora comentado

\usepackage{xfrac}
\usepackage{bbold}

\setcounter{secnumdepth}{6}

\usepackage{titlesec}
\titleformat*{\subsection}{\normalsize \bfseries}

\usepackage[colorlinks=true,linkcolor=black,urlcolor=blue,citecolor=blue]{hyperref}

%%%%%%%%%%%%%%%%%%%%%%%%%%%%%%%%%%%
%     TÍTULO, AUTORES Y FECHA              %
%%%%%%%%%%%%%%%%%%%%%%%%%%%%%%%%%%%

\title{Predicción de Ingresos: \\ Análisis del Mercado Laboral en Bogotá}

\author{
    \centering
    Harold Stiven Acuña \newline
    José David Cuervo \newline
    José David Dávila \newline
    César Augusto Alfaro
} 
\date{\today}

\begin{document}

\maketitle
\thispagestyle{empty}

%%%%%%%%%%%%%%%%%%%%%%%%%%%%%%%%%%%
%            ABSTRACT                       %
%%%%%%%%%%%%%%%%%%%%%%%%%%%%%%%%%%%

\begin{abstract}
[Pendiente: Escribir abstract de 100 palabras]
\end{abstract}

\medskip

\begin{flushleft}
    {\bf Palabras clave:} predicción de ingresos, brecha salarial, economía laboral, aprendizaje automático \\
    {\bf Clasificación JEL:} J31, C53, J16
\end{flushleft}

\pagebreak
\doublespacing

%%%%%%%%%%%%%%%%%%%%%%%%%%%%%%%%%%%
%           DOCUMENTO                       %
%%%%%%%%%%%%%%%%%%%%%%%%%%%%%%%%%%%

\section{Introducción}
[Introducción siguiendo el formato RAP: Research question, Answer, Positioning]

\section{Datos}
[Descripción de datos GEIH y proceso de web scraping]

\section{Perfil Edad-Salario}
[Análisis del perfil edad-salario]

\section{Brecha Salarial de Género}
[Análisis de la brecha salarial]

\section{Predicción de Ingresos}
[Modelos predictivos y validación]

\section{Conclusiones}
[Conclusiones principales]

%%%%%%%%%%%%%%%%%%%%%%%%%%%%%%%%%%%
%          Referencias                      %
%%%%%%%%%%%%%%%%%%%%%%%%%%%%%%%%%%%

\pagebreak
\singlespacing
\bibliography{references}
\pagebreak

%%%%%%%%%%%%%%%%%%%%%%%%%%%%%%%%%%%
%           TABLAS                          %
%%%%%%%%%%%%%%%%%%%%%%%%%%%%%%%%%%%
\section*{Tablas y Figuras}

% [Aquí se incluirán las tablas generadas por R]

\pagebreak

%%%%%%%%%%%%%%%%%%%%%%%%%%%%%%%%%%%
%           APÉNDICE                        %
%%%%%%%%%%%%%%%%%%%%%%%%%%%%%%%%%%%
\appendix
\renewcommand{\theequation}{\Alph{chapter}.\arabic{equation}}

\setcounter{figure}{0}
\setcounter{table}{0}
\makeatletter 
\renewcommand{\thefigure}{A.\@arabic\c@figure}
\renewcommand{\thetable}{A.\@arabic\c@table}

\section{Apéndice: Tablas y Figuras}\label{sec:appendix_tables} 

\end{document}
